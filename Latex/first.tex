\documentclass[a4paper,11pt,twocolumn,dvips]{report}
\usepackage[utf8]{inputenc}
\usepackage[dvips]{graphicx}
\usepackage{url}
\usepackage{cite}
\usepackage{multicol}


\setlength{\topmargin}{-0.25in}
\setlength{\headheight}{0in}
\setlength{\textheight}{9.5in}
\setlength{\headsep}{0in}
\setlength{\oddsidemargin}{-0.25in}
\setlength{\evensidemargin}{-0.25in}
\setlength{\textwidth}{7.0in}


%opening
\title{{\bf A Sample Research Paper for Demo of Latex} }
%\numberofauthors{2}
\author{1.Abhishek Pratap singh
abhi1kush@gmail.com}

\begin{document}
\maketitle
\tableofcontents
\begin{abstract}
{ \it WiFi mesh networks with outdoor links have become an at-
tractive option to provide cost-effective broadband connec-
tivity to rural areas, especially in developing regions. It is
well understood that a TDMA-based approach is necessary
to provide good performance over such networks. While pre-
liminary prototypes of TDMA-based MAC protocols have
been developed, there is no implementation-based valida-
tion/evaluation in multi-hop settings. In this work, we de-
scribe the elements of a multi-hop MAC implementation
based on the open-source MADWIFI driver. We also present
an evaluation, with a detailed accounting of the various over-
heads, on a 4-hop (5-node) path. We show that our imple-
mentation has no system overheads, achieves good through-
put, and low delay/jitter. Although the significance of TDMA is recognized, its prac-
ticality has always been in question, especially in multi-
hop settings. Can effective time-synchronization really be
achieved? What would be the overheads in practice, withelf WiFi hardware. We present some of these which
have demonstrated TDMA implementations.
SoftMAC  explores the use of the MADWIFI driver for
Atheros-based WiFi radios to experiment with MAC proto-
cols. 
FreeMAC also implements channel switching in the TDMA
system but does not implement multiple hops. FreeMAC
uses the hardware beacon timer and indicates that the timer
works well under both low load and heavy load conditions.}
\end{abstract}

\section{Introduction}
\label{sec-intro}
WiFi-based outdoor mesh networks have been demonstrated
to be a viable option for providing cost-effective broadband
connectivity to rural regions [1, 2, 3]. It is also well known
that a TDMA-based MAC is necessary for good performance
in such networks [1, 2, 4], with the default CSMA/CA oper-
ation being inefficient.
Although the significance of TDMA is recognized, its prac-
\begin{multicols}{2}ticality has always been in question, especially in multi-
hop settings. Can effective time-synchronization really be
achieved? What would be the overheads in practice, with
multiple hops? Can an implementation work using low-cost
off-the-shelf hardware? Can system overheads be controlled
so that the wireless capacity can be maximized? These are
significant questions can only be answered through a pro-
totype.
\end{multicols}
However, prototype-based evaluations of multi-hop
wireless TDMA systems have been few and far between.
According to \cite{sampe1}
In this paper, we demonstrate a TDMA implementation
that can be used over outdoor multi-hop networks using
off the shelf inexpensive hardware. Our implementation
is based on the open-source MADWIFI driver. It includes
multi-hop synchronization, and schedule dissemination from

\begin{table*}
\label{tab-table}
\centering
\begin{tabular}{|c|c|c|c|c|c|c|c|c|c|c|c|}
\hline
First & second & Third & Fourth& Fifth & Sixth & Seventh & Eighth&Nineth&Tenth\\
\hline
4 & 56 & 67 &89&4 & 56 & 67 &89&4 & 56 \\
\hline
4 & 56 & 67 &89&4 & 56 &345& 54& 556& 865\\
\hline
4 & 56 & 67 &89&4 & 56 &56& 34& 34&90\\
\hline
4 & 56 & 67 &89&4 & 56 &34& 788& 324&82\\
\hline

\end{tabular}
\end{table*}

There has been considerable effort in the area of driver-
level radio configuration using open source drivers which fa-
cilitates implementing various protocols over inexpensive off-
the-shelf WiFi hardware. We present some of these which
have demonstrated TDMA implementations.
{\huge Refer}\ref{sec-intro}
SoftMAC  explores the use of the MADWIFI driver for
Atheros-based WiFi radios to experiment with MAC proto-
cols. It disables RTS/CTS, MAC level ACKs, and facilitates\cite{article1}
custom packet header formats by setting the card in moni-
tor mode. To demonstrate the utility of the platform,  has
implemented a TDMA system between two nodes. For our
work, we borrow\cite{sample5} from SoftMAC, insights about disabling
certain aspects of CSMA.
\footnote{ For our
work, we borrow from SoftMAC, insights about disabling
certain aspects of CSMA}
\section{Related Work}
MadMAC  also implements an example TDMA sys-
tem between two machines with slot sizes of 20ms-60ms and
guard bands of 4ms-12ms. However, since we envision a mul-
{\bf\it \large  Refer to Table }\ref{tab-table}tihop system, increased slot size has a detrimental effect on
the delay/jitter. We have used slot sizes and guard bands
much smaller than those in MadMAC.
Building over SoftMAC, FreeMAC  exposes many more
configurable parameters. It also demonstrates a TDMA sys-
tem, but it uses out-of-band ethernet for synchronization.
FreeMAC also implements channel switching in the TDMA
system but does not implement multiple hops. FreeMAC
uses the hardware beacon timer and indicates that the timer
works well under both low load and heavy load conditions.\cite{google}
However, we found that the hardware timer is very sloppy
with an increased number of RX interrupts as described in
\begin{equation}
z=\left(\frac{d}{dx}\left(\frac{z^y}{\log y}\right),\int_{x=-\infty}^{y=\infty}\frac{\log \tan \theta}{\sqrt{\log \cos \theta}}\right)_{max}
\end{equation}
Sec. 4.5. FreeMAC’s insights into various aspects of Mad-
Wifi, including the hardware timer, has served as a starting
point for our work.
Overlay MAC uses the Click router system and imple-
ments a configurable module between the MAC layer and

the network layer.

 However, unlike our work it does not
have precise control over packet transmission times. And
it implements a distributed algorithm for allocating slots
are loss-based, e.g. TCP-Reno. In this situation, the delay-based
approaches may suffer from significant lower throughput than
their fair share. The reason is that delay-based approaches re-
duce their sending rate when bottleneck queue is built to avoid\cite{wiki}
self-induced packet losses. However, this behavior will encour-
age loss-based flows to increase their sending rate as they may
observe less packet losses. As a consequence, the loss-based
flows will obtain much more bandwidth than their share and
delay-based flows may even be starved 
In this paper, we propose a new congestion control protocol
for high-speed and long delay environment that satisfies all
aforementioned three requirements. Our new protocol is a syn-
ergy of both delay-based and loss-based congestion avoidance
approaches, which we call it Compound TCP (CTCP).

\section{The key Features of TCP}
\label{sec-section}

idea of CTCP is to add a scalable delay-based component to
standard TCP 1 .
\begin{itemize}
 \item {\bf List of network protocols}
 \item Bluetooth protocols
 \item Fibre Channels network protocols
 \begin{itemize}
 \item{\bf Network and Transport layer protocol}
 \item Internet Protocol Suite or TCP/IP model 
 \item OSI protocols
 \item Routing protocols
  \end{itemize}
 \begin{enumerate}
 \item {\bf Application layer Protocol}
 \item SSH
 \item FTP
 \item SMTP
 \item Telnet
 \end{enumerate}
 
 \end{itemize} This delay-based component has a scalable
window increasing rule that not only can efficiently use the link
capacity, but can also react early to congestion by sensing the
changes in RTT. If a bottleneck queue is sensed, the delay-
based component gracefully reduces the sending rate. This way,
CTCP achieves good RTT fairness and TCP fairness.
\subsection{Analysis and ex-
perimental Results }
We have
developed analytical model of CTCP and performed compre-
hensive performance studies on CTCP based on our implemen-
tation on Microsoft Windows platform. Our analysis and ex-
perimental results suggest that CTCP is a promising algorithm
to achieve high link utilization and while maintaining
Although\cite{sample6} the significance of TDMA is recognized, its prac-
ticality has always been in question, especially in multi-
hop settings. Can effective time-synchronization really be
achieved? What would be the overheads in practice, with
multiple hops? Can an implementation work using low-cost
off-the-shelf hardware? Can system overheads be controlled
so that the wireless capacity can be maximized? These are
significant questions can only be answered through a pro-
totype. However, prototype-based evaluations of multi-hop\cite{sample2}
wireless TDMA systems have been few and far between.
In this paper, we demonstrate a TDMA implementation
that can be used over outdoor multi-hop networks using
off the shelf inexpensive hardware. Our implementation
is based on the open-source MADWIFI driver.
\begin{figure*}
\label{fig-circles}
\centering
\includegraphics[scale =0.3]{drawing.eps}
\end{figure*}
\input{latech}
It includes
multi-hop synchronization, and schedule dissemination from
There has been considerable effort in the area of driver-
level radio configuration using open source drivers which fa-

cilitates implementing various protocols over inexpensive off-
the-shelf WiFi hardware. We present some of these which
have demonstrated TDMA implementations.
SoftMAC  explores the use of the MADWIFI driver for
Atheros-based WiFi radios to experiment with MAC proto-
cols. It disables RTS/CTS, MAC level ACKs, and facilitates
custom packet header formats by setting the card in moni-
tor mode. To demonstrate the utility of the platform,  has\cite{sample3}
implemented a TDMA system between two nodes. For our
work, we borrow from SoftMAC, insights about disabling
certain aspects of CSMA.
\footnote{ For our
work, we borrow from SoftMAC, insights about disabling
certain aspects of CSMA}
MadMAC  also implements an example TDMA sys-
\begin{equation}
\label{eq-equation1}
\lim_{h \to +\infty} \frac{h\sin\theta}{\log \sec h\theta}
\end{equation}
tem between two machines with slot sizes of 20ms-60ms and

{\bf\it Refer to section }\ref{sec-section}
guard bands of 4ms-12ms. However, since we envision a mul-
tihop system, increased slot size has a detrimental effect on
the delay/jitter. We have used slot sizes and guard bands
A graph of the highest TCP seqeunce number sent by node0 and the
    highest ACK received by node0 as a function of time (time on x-axis)
    . This is called the TCP time seqeunce graph. You will have one point
    on the graph for every data packet sent/ack received.
   {\bf\it Refer to figure }\ref{fig-circles} A graph of the average TCP throughput seen by node1 as a function of time. You must compute throughput over windows of 1 second. You will have one point on the graph for every second.
    A graph of the evolution of cwnd at node0 as a function of time.
    A graph of the queue occupancy (that is, the number of packets enqueued but not yet dequeued) at node 0 as a function of time.

\begin{equation}
 y= \int_{x=-\infty}^{y=\infty}\frac{\sin A}{\sqrt{\log \cos B}}
\end{equation}    A graph of the queueing delays experienced by packets as a function of time. Every time a packet is dequeued at node0, compute the amount of time it spent in the queue, and plot a point on the graph for every such sample. 
You should automate the entire process of generating graphs using a scripting language of your choice. You should have a top-level script called, say, "pa2-part1.sh" (it can be a bash/python/perl/any other script). This script should take as input a folder containing the output files from the simulation (pcaps, cwnd and ascii traces). {\bf\it Refer to equation }\ref{eq-equation}It should produce the above graphs (eps/png/jpg/pdf) into the same folder. For example, suppose we have run the simulation and moved all the pcap and trace output files into a folder called "/home/cs641/pa2-part1-analysis". Then, your script should be invoked as shown 
much smaller than those in MadMAC.
Building over SoftMAC, FreeMAC  exposes many moren nconfigurable parameters. It also demonstrates a TDMA sys-
tem, but it uses out-of-band ethernet for synchronization.
FreeMAC also implements channel switching in the TDMA
system but does not implement multiple hops. FreeMAC\cite{sample4}
uses the hardware beacon timer and indicates that the timer
works well under both low load and heavy load conditions.
However, we foun
\footnote{This is footnote blah blah balah}
$$x^7+x^2-7px+3q $$

\input{latech3}
\bibliographystyle{plain}
\bibliography{reference}

\end{document}
