\chapter{Background}
\label{ch:bg}       
\subsection{Bell Lapadula security model \cite{lapadula}}
This model defines four senstivity labels "Top secret", "secret", "classified", and "Public" each object must be labled from one of these labels. This model uses mandetory access control , mandetory denotes that they can not be changed. System contains subjects (user) , labeled objects, state machine with a set of states it allowed to go. Model preserves security of information in transitions of one state to another. Introduces (i) *(star) prperty: No Write-Down (NWD), it prevents subjects from higher label to write in to objects of lower label, this stops leak of information. (ii) simple security property: No Read Up (NRU), it prevents subjects from lower label to read from objects from higher label. A access matrix of subject and object is used to define permissions for subjects to use objects. This model is based on confidentiality of information.
\subsection{Biba Security Model \cite{biba}}  
This model ensures integrity of data. To maintain integrity of data it ensures three things (a) Prevention of modification from unathorized user. (b) Prevention of unathorized modification from authorized user. Bia model defines integrity classes and rules to preserve integrity of data. The simple integrity rule says that subjects can not read objects of lower label of integrity (No read down). The *(star) integrity rules says that subjects can not write into objects of higher label(No write up). These rules are in contrast with Bell Lapadula model because Biba model considers integrity of information instead of confidentiality. Here labels denotes degree of trust. Lowest label is not reliable so it is not allowed to write to others similarly highest label is not allowed to read from others otherwise it can be corrupted by unreliable information.
\subsection{Denning's Lattice model \cite{denning}}
This model is based on Lattice model. There is set of security classes each denotes disjoint set of information, unlike previous models it gives facility to perform operation on two security class labels, operations are lub denoted by $\oplus$ and glb denoted by $\otimes$. Both operation helps to calculate security label of an expression involving many objects from different security classes. To preserve security in information flow there is a basic rule: if information flowing from x to y (x\marr y) then for secure information flow constraint \dud{x}$\le$\dud{y} must satisfy, \dud{x} denotes security class of x. So all information flow heading upward in lattice diagram are secure.  
\subsection{Reader Writer Flow Model \cite{rwfm}} 
---


